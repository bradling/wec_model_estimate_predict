\documentclass[12pt]{article}

% Use 1 1/2 line spacing
%\usepackage{setspace}
%\setstretch{1.3}
% Set margins to 1.0 in
%\usepackage[margin=1.0in]{geometry}

% use amsmath package for cases
\usepackage{amsmath}
\usepackage{multirow}

% use this package for including graphics
%\usepackage{graphicx}

%opening
\title{WEC Time Domain Modeling Notes}
\author{Bradley Ling}

\begin{document}

\maketitle

\section{Time Domain Model}

The time domain equations of motion for a heaving body are given by
\[
	m \ddot{z} = F_r + F_e + F_k + F_{\text{PTO}} + F_{\text{mooring}} + F_{\nu}.
\]
In this analysis we neglect mooring forces $F_{\text{mooring}}$ and viscous forces $F_{\nu}$, yielding
\[
	m \ddot{z} = F_r + F_e + F_k + F_{\text{PTO}},
\]
where $m$ is the dry mass, $z$ is the heave position, $F_r$ is the radiation force, $F_e$ is the excitation force, $F_k$ is the hydrostatic force, and $F_{\text{PTO}}$ is the force of the power takeoff device on the body.

The radiation force in the time domain can be calculated from the frequency response, by convolving the frequency response with the velocity of the body,
\[
	\mathbf{F}_r(\omega) = [\mathbf{R}(\omega) + i \omega \mathbf{A}(\omega)] \mathbf{\dot{z}}(\omega),
\]
where $\mathbf{R}(\omega)$ is the frequency-dependent radiation resistance, and $\mathbf{A}(\omega)$ is the added mass. Both can be calculated with a commercial hydrodynamic package such as ANSYS Aqwa.

Using the Kramers-Kronig relationship, this can be reduced to
\[
	F_r(t) = - k(t) * \dot{z}(t) - \mathbf{A}(\infty) \ddot{z}(t),
\]
where
\[
	-k(t) = \frac{2}{\pi} \int_0^\infty \mathbf{R}(\omega) \cos(\omega t)  \, d \omega
\]
is the impulse response function of the radiation force.
Defining
\begin{align*}
	F_r^\prime(t) &= - k(t) * \dot{z}(t) \\
	&=  - \int_{-\infty}^t  k(t-\tau) \dot{z}(\tau) \, d \tau
\end{align*}
we can rewrite the time-domain equations of motion as
\[
	m \ddot{z} = F_r^\prime(t) - \mathbf{A}(\infty) \ddot{z} + F_e + F_k + F_{\text{PTO}},
\]
or
\[
	[m + \mathbf{A}(\infty)] \ddot{z} = F_r^\prime + F_e + F_k + F_{\text{PTO}}.
\]


\subsection*{Model Verification}

I Still need to figure out exactly how to verify my model.
Right now what I am doing is the following
Compare
\[
	\max (F_r^\prime (t)) \quad \text{for }\eta\text{ at frequency }\omega
\]
and
\[
	\big|\big| \, k(\omega) \max( \, \dot{z}(t) \, ) \, \big|\big|
\]

But it doesnt seem to be working properly. See Falnes Ocean Waves and Oscillating Systems to figure it out.


\section*{State Space Approximation of Raidiation Force}

To model the radiation force with a reduced order state space model, we first assume the radiation force can be modeled with the following form
\[
	\dot{\zeta}_r = A_R \zeta_r + B_R \dot{z}
\]
\[
	F_r^\prime = 
	\begin{bmatrix}
		0 & 0 & \cdots & 0 & 1
	\end{bmatrix} 
	\zeta_r.
\]
The input to this dynamic system is the heave velocity, and the output of the system is the radiation force $F_r^\prime$. The challenge then is to determine what $A_R$ and $B_R$ must be to closely approximate the frequency response calculated with ANSYS Aqwa.

This can be done by assuming a form for $A_R$ and $B_R$ and utilizing system identification techniques. Noting that the impulse response for the state space model ($k_{SS}(t)$) is given by
\[
	k_{SS}(t) = C_R e^{t A_R} B_R.
\]
If we force the state space model to be in companion form
\[
	A_R = 
	\begin{bmatrix}
		0 & 0 & 0 & \cdots & 0 & -a_1 \\
		1 & 0 & 0 & \cdots & 0 & -a_2 \\
		0 & 1 & 0 & \cdots & 0 & -a_3 \\
		\vdots & \vdots & \vdots & \ddots & \vdots & \vdots \\
		0 & 0 & 0 & \cdots & 1 & -a_n
	\end{bmatrix},
\]
\[
	B_R = 
	\begin{bmatrix}
		b_1 & b_2 & \cdots & b_n
	\end{bmatrix}^T,
\]
then we can determine the best response by optimizing $\underline{a} = [-a_1 \, \cdots  \, -a_n]^T$ and $B_R$ to minimize the error in the impulse response function. This optimization has $2n$ degrees of freedom, where $n$ is the order of the state space model. Given $m$ discrete values of $k(t)$, the unconstrained optimization problem becomes
\[
	\min \; Q(\underline{a}, B_R) = \sum_{p = 1}^{m} G(p) [k(p) - C_R e^{t_p A_R} B_R ]^2,
\]
where $G(p)$ is an optional weighting function. I found a good local minima using Matlab's Otpimization toolbox \texttt{fminunc} function.

With this approximation a full state space model can now be written to simulate the movement of the body. First the state vector is defined as
\[
	\xi = \begin{bmatrix}
		\zeta_r \\
		z \\
		\dot{z}
	\end{bmatrix},
\]
the full state space system model can be written as
\[
\dot{\xi} = 
\begin{bmatrix}
	 & & & 0 &  \\
	& A_R & & \vdots & B_R \\
	 & & & 0 &  \\
	 0 & \cdots & 0 & 0 & 1 \\
	 0 & \cdots & \frac{1}{m_{\text{tot}} } & \frac{K_{\text{hyd}}}{m_{\text{tot}} } & 0
\end{bmatrix}
\xi
+ 
\begin{bmatrix}
	0 \\
	\vdots \\
	0 \\
	0 \\
	\frac{1}{m_{\text{tot}} }
\end{bmatrix}
f_e(t)
\]
where
\[
	m_{\text{tot}} = m + \mathbf{A}(\infty).
\]


\end{document}
